\documentclass[12pt,a4paper]{article}
\usepackage[utf8]{inputenc}
\usepackage[english]{babel}
\usepackage{amsmath}
\usepackage{amssymb}
\usepackage{graphicx}
\usepackage{listings}
\usepackage{xcolor}
\usepackage{hyperref}
\usepackage{geometry}
\geometry{margin=1in}

% Python code styling
\definecolor{codegreen}{rgb}{0,0.6,0}
\definecolor{codegray}{rgb}{0.5,0.5,0.5}
\definecolor{codepurple}{rgb}{0.58,0,0.82}
\definecolor{backcolour}{rgb}{0.95,0.95,0.92}

\lstdefinestyle{pythonstyle}{
    backgroundcolor=\color{backcolour},   
    commentstyle=\color{codegreen},
    keywordstyle=\color{magenta},
    numberstyle=\tiny\color{codegray},
    stringstyle=\color{codepurple},
    basicstyle=\ttfamily\footnotesize,
    breakatwhitespace=false,         
    breaklines=true,                 
    captionpos=b,                    
    keepspaces=true,                 
    numbers=left,                    
    numbersep=5pt,                  
    showspaces=false,                
    showstringspaces=false,
    showtabs=false,                  
    tabsize=2,
    language=Python
}

\lstset{style=pythonstyle}

\title{Session 1: 1D Ordinary Differential Equations\\
\large Building an Interactive Spruce Budworm Simulation}
\author{Applied Math Modeling Lab}
\date{23rd January 2026}

\begin{document}

\maketitle

\section{Introduction}

Welcome to this hands-on session on modeling population dynamics using ordinary differential equations (ODEs). Today, we will explore the \textbf{spruce budworm model}, a classic example from ecological modeling that demonstrates how simple nonlinear systems can exhibit complex behaviors including multiple equilibria and catastrophic transitions.

\subsection{Learning Objectives}

By the end of this session, you will be able to:
\begin{itemize}
    \item Understand the mathematical formulation of the spruce budworm model
    \item Implement the model as a Python function
    \item Solve the ODE numerically using \texttt{scipy.integrate.solve\_ivp}
    \item Visualize the phase portrait and identify equilibrium points
    \item Build an interactive Streamlit application to explore the model
\end{itemize}

\subsection{The Spruce Budworm Model}

The spruce budworm is an insect that periodically devastates spruce forests. The population dynamics can be modeled by the following ODE:

\begin{equation}
\frac{dx}{dt} = rx\left(1 - \frac{x}{k}\right) - \frac{x^2}{1 + x^2}
\end{equation}

where:
\begin{itemize}
    \item $x(t)$ is the budworm population (adimensional)
    \item $r$ is the intrinsic growth rate (typically $r \approx 0.5$)
    \item $k$ is the carrying capacity of the forest (typically $k \approx 10$)
\end{itemize}

The first term represents logistic growth, while the second term models predation by birds (which follows a saturating functional response).

\textbf{Reference:} Strogatz, S. H. (2018). \textit{Nonlinear Dynamics and Chaos}, Chapter 3.7.

%--------------------------------------------------------------

\section{Implementing the ODE Function}

\subsection{Task}
Create a Python function that implements the spruce budworm differential equation. The function should follow the signature required by \texttt{scipy.integrate.solve\_ivp}.

\subsection{Requirements}
\begin{itemize}
    \item Function name: \texttt{spruce\_budworm}
    \item Parameters: \texttt{t} (time), \texttt{x} (population), \texttt{r} (growth rate), \texttt{k} (carrying capacity)
    \item Return: The rate of change $\frac{dx}{dt}$
    \item Include appropriate docstring documentation
\end{itemize}

\subsection{Hint}
The function signature should be:
\begin{lstlisting}
def spruce_budworm(t: float, x: float, r: float = 0.5, k: float = 10) -> float:
    """Model for the spruce budworm population dynamics."""
    # Your implementation here
    dxdt = ???
    return dxdt
\end{lstlisting}

\subsection{Mathematical Expression}
Remember to implement:
\begin{equation}
\frac{dx}{dt} = rx\left(1 - \frac{x}{k}\right) - \frac{x^2}{1 + x^2}
\end{equation}

%--------------------------------------------------------------

\section{Phase Portrait Visualization}

\subsection{Task}
Create a function that plots the rate of change $\frac{dx}{dt}$ as a function of the population $x$. This phase portrait will help us visualize the equilibrium points and their stability.

\subsection{Requirements}
\begin{itemize}
    \item Function name: \texttt{plot\_spruce\_budworm\_rate}
    \item Plot $\frac{dx}{dt}$ vs $x$ for $x \in [0, k]$
    \item Identify and mark equilibrium points (where $\frac{dx}{dt} = 0$)
    \item Color-code equilibrium points:
    \begin{itemize}
        \item Blue circles for stable equilibria (where $\frac{dx}{dt}$ crosses zero from above)
        \item Red circles for unstable equilibria (where $\frac{dx}{dt}$ crosses zero from below)
    \end{itemize}
    \item Add a horizontal line at $y = 0$
    \item Mark the current population $x_t$ with a vertical dashed line
\end{itemize}

\subsection{Hints}
\begin{itemize}
    \item Use \texttt{np.linspace} to create an array of $x$ values
    \item To find zero crossings: look for sign changes using \texttt{np.diff(np.sign(dxdt))}
    \item Stability: a fixed point is stable if $\frac{dx}{dt}$ decreases as you pass through it
\end{itemize}

\subsection{Key Concepts}
\begin{itemize}
    \item \textbf{Equilibrium point}: A value $x^*$ where $\frac{dx}{dt} = 0$
    \item \textbf{Stable equilibrium}: Small perturbations decay back to $x^*$
    \item \textbf{Unstable equilibrium}: Small perturbations grow away from $x^*$
\end{itemize}

\section{Numerical Integration}

\subsection{Task}
Create a function that evolves the system forward in time using numerical integration. This function should solve the ODE and append the results to existing time and population arrays.

\subsection{Requirements}
\begin{itemize}
    \item Function name: \texttt{evolve\_spruce\_budworm}
    \item Use \texttt{scipy.integrate.solve\_ivp} with the RK45 method
    \item Parameters:
    \begin{itemize}
        \item \texttt{t}: existing time array
        \item \texttt{x}: existing population array
        \item \texttt{r}, \texttt{k}: model parameters
        \item \texttt{t\_eval}: duration to evolve forward
    \end{itemize}
    \item Start from the last values in the input arrays
    \item Concatenate new results to the input arrays
    \item Ensure population never goes negative (use \texttt{np.clip})
    \item Return updated time and population arrays
\end{itemize}

\subsection{Code Structure}
\begin{lstlisting}
def evolve_spruce_budworm(
    t: np.ndarray, x: np.ndarray, 
    r: float = 0.5, k: float = 10, 
    t_eval: float = 50
) -> tuple[np.ndarray, np.ndarray]:
    # Define time span from last time point
    t_span = (t[-1], t[-1] + t_eval)
    
    # Create evaluation points
    t_eval_points = np.linspace(???)
    
    # Solve the ODE
    solution = solve_ivp(???)
    
    # Concatenate results
    t = np.concatenate((t, solution.t))
    x = np.concatenate((x, solution.y[0]))
    
    # Ensure non-negative population
    x = np.clip(x, a_min=0, a_max=None)
    
    return t, x
\end{lstlisting}

\subsection{Important Notes}
\begin{itemize}
    \item The \texttt{args} parameter in \texttt{solve\_ivp} passes additional arguments to your ODE function
    \item Initial condition should be \texttt{[x[-1]]} (last population value)
    \item Use \texttt{method="RK45"} for adaptive step-size Runge-Kutta integration
\end{itemize}

%--------------------------------------------------------------

\section{Time Series Visualization}

\subsection{Task}
Create a function to plot the population dynamics over time. This visualization shows how the population evolves from the initial condition.

\subsection{Requirements}
\begin{itemize}
    \item Function name: \texttt{plot\_spruce\_budworm}
    \item Plot time on the x-axis and population on the y-axis
    \item Use green color for the trajectory
    \item Ensure y-axis starts at 0 (populations cannot be negative)
    \item Include grid, labels, and title
    \item Return the figure and axes objects
\end{itemize}

\subsection{Questions to Consider}
\begin{itemize}
    \item What happens to the population in the long run?
    \item Does the final state depend on the initial condition?
    \item How do different parameter values affect the dynamics?
\end{itemize}

%--------------------------------------------------------------

\section{Building the Streamlit Application}

\subsection{Task}
Now that you have all the components, create an interactive Streamlit application that allows users to explore the spruce budworm model. This will enable real-time parameter adjustment and visualization.

\subsection{Requirements}

\subsubsection{Sidebar Controls}
Create sliders for:
\begin{itemize}
    \item Growth rate $r$ (range: 0.1 to 1.0)
    \item Carrying capacity $k$ (range: 5 to 20)
    \item Initial population $x_0$ (range: 0.1 to $k$)
    \item Evolution time step (default: 50)
\end{itemize}

\subsubsection{Interactive Features}
\begin{itemize}
    \item Display the differential equation with current parameter values
    \item Show the phase portrait (rate of change plot)
    \item Show the time series evolution
    \item Add a button to "Evolve Forward" that continues the simulation
    \item Use \texttt{st.session\_state} to maintain simulation state between button clicks
\end{itemize}

\subsubsection{Layout Structure}
\begin{lstlisting}
import streamlit as st
import numpy as np
import matplotlib.pyplot as plt
from your_module import (
    spruce_budworm,
    plot_spruce_budworm_rate,
    evolve_spruce_budworm,
    plot_spruce_budworm
)

st.title("Spruce Budworm Population Dynamics")

# Sidebar parameters
r = st.sidebar.slider("Growth rate (r)", ...)
k = st.sidebar.slider("Carrying capacity (k)", ...)
x0 = st.sidebar.slider("Initial population", ...)

# Initialize session state
if 'initialized' not in st.session_state:
    st.session_state.t = np.array([0])
    st.session_state.x = np.array([x0])
    st.session_state.initialized = True

# Display equation
st.latex(r"\frac{dx}{dt} = rx\left(1 - \frac{x}{k}\right) - \frac{x^2}{1 + x^2}")

# Plot phase portrait
fig1, ax1 = plot_spruce_budworm_rate(...)
st.pyplot(fig1)

# Plot time series
fig2, ax2 = plot_spruce_budworm(...)
st.pyplot(fig2)

# Evolution button
if st.button("Evolve Forward"):
    st.session_state.t, st.session_state.x = evolve_spruce_budworm(...)
    st.rerun()
\end{lstlisting}

\subsection{Advanced Features (Optional)}
\begin{itemize}
    \item Add a reset button to restart the simulation
    \item Display current population value and equilibrium states
    \item Add explanatory text about bifurcations and catastrophes
    \item Show multiple trajectories with different initial conditions
    \item Add animation of the population dynamics
\end{itemize}

%--------------------------------------------------------------

\section{Exploration Questions}

Once your simulation is working, explore the following questions:

\begin{enumerate}
    \item \textbf{Multiple Equilibria:} For $r = 0.5$ and $k = 10$, how many equilibrium points exist? Which are stable?
    
    \item \textbf{Bistability:} Start with two different initial conditions (e.g., $x_0 = 1$ and $x_0 = 8$). Do they converge to the same equilibrium?
    
    \item \textbf{Hysteresis:} Slowly increase the carrying capacity $k$ from 5 to 15. Then slowly decrease it back to 5. Does the population return to the same state?
    
    \item \textbf{Outbreak Dynamics:} What happens if you start with a small population ($x_0 < 2$) and the carrying capacity is large ($k > 10$)?
    
    \item \textbf{Critical Slowing Down:} When the population is near an unstable equilibrium, how long does it take to move away? Compare this to the rate of change far from equilibrium.
    
    \item \textbf{Parameter Space:} Create a diagram showing the number of equilibria as a function of $r$ and $k$. Where do bifurcations occur?
\end{enumerate}

\section{Mathematical Background}

\subsection{Equilibrium Analysis}

Equilibrium points satisfy:
\begin{equation}
rx^*\left(1 - \frac{x^*}{k}\right) - \frac{(x^*)^2}{1 + (x^*)^2} = 0
\end{equation}

This can be rewritten as:
\begin{equation}
rx^*\left(1 - \frac{x^*}{k}\right) = \frac{(x^*)^2}{1 + (x^*)^2}
\end{equation}

The left side represents birth rate (logistic growth), and the right side represents predation rate. Equilibria occur where these balance.

\subsection{Stability Analysis}

The stability of an equilibrium $x^*$ is determined by the sign of the derivative:
\begin{equation}
\frac{d}{dx}\left(\frac{dx}{dt}\right)\bigg|_{x=x^*}
\end{equation}

If this derivative is:
\begin{itemize}
    \item Negative: the equilibrium is stable (attracting)
    \item Positive: the equilibrium is unstable (repelling)
    \item Zero: higher-order analysis is needed
\end{itemize}

\subsection{Ecological Interpretation}

\begin{itemize}
    \item \textbf{Low equilibrium:} Few budworms, controlled by predation
    \item \textbf{High equilibrium:} Outbreak state, budworms overwhelm predators
    \item \textbf{Middle equilibrium:} Usually unstable, separates the two basins of attraction
    \item \textbf{Hysteresis:} The system can "jump" between states depending on history
\end{itemize}

This behavior explains why spruce budworm populations can suddenly explode from low levels to outbreak proportions, and why simply reducing the outbreak may not return the forest to a healthy state.

%--------------------------------------------------------------

\section{Resources}

\begin{itemize}
    \item Strogatz, S. H. (2018). \textit{Nonlinear Dynamics and Chaos}. CRC Press, Chapter 3.7
    \item SciPy documentation: \url{https://docs.scipy.org/doc/scipy/reference/generated/scipy.integrate.solve_ivp.html}
    \item Streamlit documentation: \url{https://docs.streamlit.io}
    \item Reference implementation: \url{https://github.com/daniprec/BAM-Applied-Math-Lab/tree/main/sessions/s01_odes_1d}
\end{itemize}

\section{Tips for Success}

\begin{itemize}
    \item \textbf{Start simple:} Get section 2 working first, then build up.
    \item \textbf{Test incrementally:} Verify each function works before moving to the next.
    \item \textbf{Use the reference:} The provided code is there to help you understand the structure.
    \item \textbf{Experiment:} Try different parameter values and see what happens.
    \item \textbf{Collaborate:} Discuss with your classmates, but write your own code.
    \item \textbf{Ask questions:} If you're stuck, ask for help!
\end{itemize}

\vspace{1cm}

\noindent\textbf{Good luck and enjoy your coding!}

\end{document}
